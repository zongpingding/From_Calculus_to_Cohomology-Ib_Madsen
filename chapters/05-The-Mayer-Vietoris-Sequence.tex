\chapter{The Mayer-Vietoris Sequence}
This chapter introduces a fundamental calculational technique for de Rham
cohomology, namely the so-called Mayer-Vietoris sequence, which calculates
$H^*(U_1\cup U_2)$ as a "function" of $H^*(U_1), H^*(U_2)$ and $H^*(U_1\cap U_2)$. Here UI and
$U_2$ are open sets in $\RR^n$. By iteration we get a calculation of $H^*(U_1\cap\cdots\cap U_n)$ as
a "function" of $H^*(u_\alpha)$, where 0: runs over the subsets of ${1,\cdots, n}$ and 
$U_{i_1}\cap\cdots\cap U_{i_r}$ when $\alpha = {i_1,\cdots, i_r }$. Combined with the Poincar\'e lemma, this
yields a \Index{principal} calculation of $H^*(U)$ for quite general open sets in $\RR^n$. If, for
instance, $U$ can be covered by a finite number of convex open sets $U_i$, then every
$U_\alpha$ will also be convex and $H^*(U_\alpha)$ thus known from the Poincar\'e lemma


\begin{theorem}\label{theorem:5-1}
Let $U_1$ and $U_2$ be open sets in $\RR^n$ with union $U = U_1\cup U_2$. For $\nu = 1, 2$, let 
$i_\nu: U_\nu\to U$ and $j_\nu:U_1\cap U_2\to U_\nu$ be the corresponding inclusions. Then the sequence
\begin{align*}
  0\xra[]\Omega^p(U)\xra[I^p]\Omega^p(U_1)\oplus\Omega^p(U_2)\xra[J^p]\Omega^p(U_1\cap U_2)\xra[]\xra[]0
\end{align*}

is exact, where $I^p(\omega) = (i^*_1(\omega), i^*_2(\omega)), J^p(\omega_1, \omega_2) = j^*_1(\omega_1) - j^*_2(\omega_2)$.
\end{theorem}

\begin{proof}
  For a smooth map $\phi:V\to W$ and a $p$-form $\omega = \sum f_I\dd x_I\in \Omega^p(W)$,
  \begin{align*}
    \Omega^p(\phi)(\omega)=\phi^*(\omega)=\sum (f_I\circ\phi)\dd\phi_{i_1}\wedge\ldots\wedge\dd\phi_{i_p}.    
  \end{align*}

  In particular, if $\phi$ is an inclusion of open sets in $\RR^n$, i.e., $\phi_i(x) = x_i$, then 
  \begin{align*}
    \dd \phi_{i_1}\wedge\cdots\wedge\dd\phi_{i_p} = \dd x_{i_1}\wedge\cdots\wedge\dd x_{i_p}.  
  \end{align*}

  Hence 
  \begin{align}\label{eq:5-1}
    \phi^*(\omega) = \sum f_I\circ\phi\dd x_I.
  \end{align}

  This will be used for $\phi = i_\nu, j_\nu, \nu=1, 2$. It follows from \eqref{eq:5-1} that $I^p$ is injective.
  If namely $I^p(\omega) = 0$ then $i^*_1(\omega) = 0 = i^*_2(\omega)$, and 
  \begin{align*}
    i^*(\nu) = \sum (f_I\circ i_\nu)\dd x_I = 0
  \end{align*}

  If and only if $f_I\circ i_\nu = 0$ for all $I$. However $f_I\circ i_1 = 0$ and $f_I\circ i_2 = 0$ imply that
  $f_I = 0$ on all of $U$, since $U_1$ and $U_2$ cover $U$. Similarly, we show that $\ker J^p = \im I^p$. First 
  \begin{align*}
    J^p\circ I^p(\omega) 
    = j^*_2i^*_2(\omega) - j^*_1i^*_1(\omega) 
    = j^*(\omega) - j^*_1(\omega) 
    = 0
  \end{align*}

  where $j:U_1\cap U_2 \to U$ is the inclusion. Hence $\im I^p \subseteq \ker J^p$. To show the converse inclusion 
  we start with two $p$-form $\omega_\nu\in \Omega^p(U_\nu)$.
  \begin{align*}
    \omega_1 = \sum f_I\dd x_I, \quad \omega_2 = \sum g_I\dd x_I.
  \end{align*}

  Since $J^p(\sum h_I\dd x_I) = (\omega_1, \omega_2)$, we have that $j_1^*(\omega_1) = j^*_2(\omega_2)$, which by 
  \eqref{eq:5-1} translates into $f_I\circ j_1 = g_I\circ j_2$ or $f_I(x) = g_I(x)$ for $x\in U_1\cap U_2$. We define 
  a smooth function $h_I:U\to \RR^n$ by
  \begin{align*}
    h_I(x) = \begin{cases}
      f_I(x), & x\in U_1,\\
      g_I(x), & x\in U_2.
    \end{cases}  
  \end{align*}
  
  Then  $I^p(\sum h_I\dd x_I) = (\omega_1, \omega_2)$. Finally we show that $J^p$ is surjective. To this end we use 
  a partition of unity $\{p-1, p_2\}$ with support in $\{U_1, U_2\}$. i.e., smooth functions $h_I:U\to \RR^n$ by 
  \begin{align*}
    p_\nu: U\to \{0, 1\}, \quad \nu = 1, 2
  \end{align*}

  for which $\R{supp}_U(p_\nu)\subset U_\nu$, and such that $p_1(x) + p_2(x) = 1$ for $x\in U$ (cf.
  Appendix A).

  Let $f:U_1\cap U_2\to\RR$ be a smooth function. We use $\{p_1, p_2\}$ to extend $f$ to $U_1$ and
  $U_2$. Since $\R{supp}_U(p_1)\cap U_2\subset U_1\cap U_2$, we can define a smooth function by
  \begin{align*}
    f_2(x) = \left\{\begin{aligned}
      & -f(x)p_1(x) && \text{ if } x\in U_1\cap U_2 \\
      & 0 && \text{ if } x\in U_2 - \R{supp}_U(p_1) 
    \end{aligned}\right.
  \end{align*}

  Analogously we define 
  \begin{align*}
    f_1(x) = \left\{\begin{aligned}
      & f(x)p_2(x) && \text{ if } x\in U_1\cap U_2 \\
      & 0 && \text{ if } x\in U_1 - \R{supp}_U(p_2) 
    \end{aligned}\right.
  \end{align*}

  Note that $f_1(x) - f_2(x) = f(x)$ when $x\in U_1\cap U_2$, because $p_1(x) + p_2(x) = 1$.
  For a differential form $\omega\in \Omega^p(U_1\cap U_2)$, $\omega = \sum f_I\dd x_I$, we can apply 
  the above to each of the functions $f_I:U_1\cap U_2\to\RR$. This yields the functions $f_{I\nu}:U_\nu\to\RR$,
  and thus the differential form $\omega_\nu = \sum f_{I, \nu}\dd x_I\in\Omega^p(U_\nu)$. With this choice $J^p(\omega_1, \omega_2) 
  = \omega$. 
\end{proof}

It is clear that
\begin{align*}
  & I{:}\Omega^{*}(U)\to\Omega^{*}(U_{1})\oplus\Omega^{*}(U_{2})\\
  & J{:}\Omega^{*}(U_{1})\oplus\Omega^{*}(U_{2})\to\Omega^{*}(U_{1}\cap U_{2})
\end{align*}

are chain maps, so that Theorem \ref{theorem:5-1} yields a short exact sequence of chain
complexes. From Theorem \ref{theorem:4-9} one thus obtains a long exact sequence of
cohomology vector spaces. Finally Lemma \ref{lemma:4-13} tells us that
\begin{align*}
  H^p(U)(\Omega^*(U_1)\oplus\Omega^*(U_2)) = H^p(U_1)\oplus H^p(U_2)
\end{align*}

We have proved:

\begin{theorem}[Mayer-Vietoris]\label{theorem:5-2}\index{sequence!Mayer-Vietoris}
Let $U_1$ and $U_2$ be open sets in $\RR^n$ and $U = U_1\cup U_2$.
There exists an exact sequence of cohomology vector spaces
\begin{align*}
  \cdots\xra[]H^p(U)\xra[I^*]H^p(U_1)\oplus H^p(U_2)\xra[J^*]H^p(U_1\cap U_2)\xra[\partial^*]H^{p+1}(U)\xra[]\cdots
\end{align*}

Here $I^*(\omega) = (i^*_1([\omega]), i^*_2([\omega]))$ and $J^*([\omega_1], [\omega_2]) = [j^*_1(\omega_1) - j^*_2(\omega_2)]$
in the notation of Theorem \ref{theorem:5-1}.
\end{theorem}

\begin{corollary}\label{corollary:5-3}
  If $U_1$ and $U_2$ are disjoint open sets in $\RR^n$ then
  \begin{align*}
    I^*:H^p(U_1\cup U_2)\to H^p(U_1)\oplus H^p(U_2)
  \end{align*}

  is an isomorphism.
\end{corollary}

\begin{proof}
  It follows from the Theorem \ref{theorem:5-1} that 
  \begin{align*}
    I^p:\Omega^p(U_1\cup U_2)\to\Omega^p(U_1)\oplus\Omega^p(U_2)
  \end{align*}

  is an isomorphism, and Lemma \ref{lemma:4-13} gives that corresponding map on cohomology is also an isomorphism.
\end{proof}

\begin{example}\label{example:5-4}
  We use Theorem \ref{theorem:5-2} to calculate the de Rham cohomology vector spaces of the punctured plane $\RR^n-\{0\}$. Let 
  \begin{align*}
    U_{1} & = \RR^2-\{(x_1,x_2)\mid x_1\geq0, x_2=0\}\\
    U_{2} & = \RR^2-\{(x_1,x_2)\mid x_1\leq0, x_2=0\}.
  \end{align*}

  These are star-shaped open sets, such that $H^p(U_1) = H^p(U_2) = 0$ for $p>0$ and $H^0(U_1) = H^0(U_2) = \RR$. 
  Their intersection 
  \begin{align*}
    U_1\cap U_2 = \RR^2-\RR = \RR^2_+\cup \RR^2_-
  \end{align*}

  is disjoint union of the open half-planes $x_2>0$ and $x_2<0$. Hence 
  \begin{align}\label{eq:5-2}
    H^p(U_1\cap U_2) = \left\{\begin{aligned}
      & 0 && \text{ if } p>0.
      & \RR\oplus\RR \text{ if} && p = 0
    \end{aligned}\right.
  \end{align}

  by the Poincar\'e lemma and Corollary \ref{corollary:5-3}. From the \Index{Mayer-Vietoris sequence} we have
  \begin{align*}
    \cdots  & \xra[] H^p(U_1)\oplus H^p(U_2)\xra[J^*]H^p(U_1\cap U_2)\xra[\partial^*]\\
            & H^{p+1}(\RR^2-\{0\})\xra[I^*] H^{p+1}(U_1)\oplus H^{p+1}(U_2)\xra[]\cdots
  \end{align*}
\end{example}

For $p>0$, 
\begin{align*}
  0\xra[] H^p(U_1\cap U_2)\xra[\partial^*] H^{p+1}(\RR^2-\{0\})\xra[] 0
\end{align*}

is exact, i.e., $\partial^*$ is an isomorphism and $H*q(\RR^2-\{0\}) = 0$ for $q>0$ according 
to \eqref{eq:5-2}.

If $p=0$, one gets the exact sequence
\begin{align}\label{eq:5-3}
\begin{aligned}
  H^{-1}(U_1\cap U_2)\xra[] H^0(\RR^2-\{0\}) \xra[I^0] H^0(U_1)\oplus H^0(U_2)\xra[J^0] \\
  H^0(U_1\cap U_2)\xra[\partial^*] H^1(\RR^2-\{0\}) \xra[I^1]H^1(U_1)\oplus H^1(U_2)
\end{aligned}
\end{align}

Since $H^{-1}(U) = 0$ for all open sets, and in particular $H^{-1}(U_\nu)$, $I^0$ is
injective. Since $H^{-1}(U_\nu) = 0$, $\partial^*$ is surjective, and the sequence \eqref{eq:5-3} reduces to
the exact sequence

\begin{align*}
  0\xra[] H^0(\RR^2-\{0\})\xra[I^0] H^0(U_1)
  \hspace*{-1em}\overset{\substack{\displaystyle\RR\otimes\RR\\\big|\!\big|}}{\oplus}\hspace*{-1em} 
  H^0(U_2)\xra[J^0] 
  \overset{\substack{\displaystyle\RR\otimes\RR\\\big|\!\big|}}{H^0(U_1\cap U_2)}
  \xra[\partial^*]H^1(\RR^2-\{0\})\xra[] 0
\end{align*}

However, $\RR^2-\{0\}$ is connected. Hence $H^0(\RR^2-\{0\}) \cong \RR$, and since $I^0$ is 
injective we must have that $\im J^0\cong\RR$. Exactness gives $\ker J^0\cong\RR$, so that $J^0$
has rank 1. Therefore $\im J^0\cong \RR$ and, once again, by exactness
\begin{align*}
  \partial^*: H^0(U_1\cap U_2)/\im J^0 \to H^1(\RR^2-\{0\})
\end{align*}

Since $H^0(U_1\cap U_2)/\im J^0\cong \RR$, we have shown
\begin{align*}
  H^p(\RR - \{0\}) = \left\{\begin{aligned}
    & 0 && \text{ if } p>2,\\
    & \RR && \text{ if } p = 1 \\
    & \RR && \text{ if } p = 0
  \end{aligned}\right.
\end{align*}

In the proof above we could alternatively have calculated
\begin{align*}
  J^0:H^0(U_1)\oplus H^0(U_2)\to H^0(U_1\cap U_2)
\end{align*}

by using Lemma \ref{lemma:3-9}: $H^0(U)$ consists of locally constant functions. If $f_i$ is a
constant function on $U_i$, then
\begin{align*}
  J^0(f_1) = f_{1|U_1\cap U_2} \text{ and } J^0(f_2) = f_{2|U_1\cap U_2}
\end{align*}

so that $J^0(a, b) = a - b$.

\begin{theorem}\label{theorem:5-5}
  Assume that the open set $U$ is covered by convex open sets $U_1, \cdots, U_r$. Then $H^p(U)$ is finitely generated.
\end{theorem}

\begin{proof}
  We use induction on the number of open sets. If $r = 1$ the assertion
follows from the Poincar\'e lemma. Assume the assertion is proved for $r - 1$ and
let $V = U_1\cup\cdots\cup U_{r-1}$, such that $U = V\cup U_r$. From Theorem \ref{theorem:5-2} we have 
the exact sequence
\begin{align*}
  H^{p-1}(V\cup U_r) \xra[\partial^*] H^p(U) \xra[I^*] H^p(V)\oplus H^p(U_r)
\end{align*}

which by Lemma \ref{lemma:4-1} yields
\begin{align*}
  H^p(U) \backsimeq \im\partial^* \oplus \ker I^*.
\end{align*}

Now both $V$ and $V\cap U_r = (U_1\cap U_r)\cup\cdots\cup (U_{r-1}\cap U_r)$ are unions by $r-1$ convex open sets.
Therefore Theorem \ref{theorem:5-5} holds for $H^*(V\cap U_r), H^*(V)$ and $H^*(U_r)$, and hence also for $H^*(U)$.
\end{proof}