\chapter{Chain Complexes and their Cohomology}
In this chapter we present some general algebraic definitions and viewpoints,
which should illuminate some of the constructions of Chapter 3. The algebraic
results will be applied later to de Rham cohomology in Chapters 5 and 6.

A sequence of vector spaces and linear maps
\begin{align}\label{eq:4-1}
  A \lr{f} B \lr{g} C
\end{align}

is said to be \Index{exact} when $\im f = \ker g$, where as above
\begin{align*}
  \ker g & = \{b\in B | g(b) = 0\}\qquad \text{the kernel of } g \\
  \im f  & = \{f(a) | a\in A\}\qquad \text{the image of } f
\end{align*}

Note that $A\lr{f}B\lr{}0$ is exact precisely when $f$ is surjective and that $0\lr{}B\lr{g}C$ is exact
precisely when $g$ is injective. A sequence $A^* = \{A^i, \dd^i\}$,
\begin{align}\label{eq:4-2}
  \cdots \lr{} A^{i-1}\lr{\dd^{i-1}} A^i \lr{\dd^i} A^{i+1} \lr{\dd^{i+1}} A^{i+2}\lr{}\cdots
\end{align}

of vector spaces and linear maps is called a \Index{chain complex} provided $\dd^{i+1}\circ\dd^i = 0$
for all $i$. It is exact if
\begin{align*}
  \ker \dd^i = \im \dd^{i-1}
\end{align*}

for all $i$. An exact sequence of the form
\begin{align}\label{eq:4-3}
  0\lr{}A\lr{f}B\lr{g}C\lr{}0
\end{align}

is called \Index{short exact}. This is equivalent to requiring that
\begin{align*}
  \text{$f$ is injective, $g$ is surjective and $\im f = \ker g$}
\end{align*}

The \Index{cokernel} of a linear map $f:A\to B$ is
\begin{align*}
  \cok(f) = B/\im(f).
\end{align*}

For a short exact sequence, $g$ induces an isomorphism
\begin{align*}
  g:\cok(f) \lr{\cong} C.
\end{align*}

Every (long) exact sequence, as in \eqref{eq:4-2}, induces short exact sequences (which can
be used to calculate $A^i$)
\begin{align*}
  0\lr{} \im\dd^{i-1} \lr{} \im\dd^i \lr{} 0
\end{align*}

Furthennore the isomorphisms
\begin{align*}
  A^{i-1}/\im\dd^{i-1} {\cong} A^{i-1}/\ker\dd^{i-1} \lr[\cong]{\dd^{i-1}} \im\dd^{i-1}
\end{align*}

are frequently applied in concrete calculations.

The direct sum of vector spaces A and B is the vector space
\begin{align*}
   & A\oplus B  = \{(a,b) | a\in A, b\in B\}                       \\
   & \lambda(a, b)  = (\lambda a, \lambda b), \qquad \lambda\in\RR \\
   & (a_1, b_1) + (a_2, b_2)  = (a_1 + a_2, b_1 + b_2)
\end{align*}

If $\{a_i\}$ and $\{b_j\}$ are bases of $A$ and $B$, respectively, then $\{(a_i, 0), (0, b_j)\}$ is a
basis of $A\oplus B$. In particular
\begin{align*}
  \dim(A\oplus B) = \dim A + \dim B
\end{align*}

\begin{lemma}\label{lemma:4-1}
  Suppose $0\lr{}A\lr{f}B\lr{g}C\lr{}0$ is a short exact sequence ofvector
  spaces. Then $B$ is finite-dimensional if both $A$ and $C$ are, and $B\cong A\oplus C$.
\end{lemma}

\begin{proof}
  Choose a basis $\{a_i\}$ of $A$ and $\{c_j\}$ of $C$. Since $g$ is surjective there
  exist $b_j\in B$ with $g(b_j) = c_j$. Then $\{f(a_i), b_j\}$ is a basis of $B$: For $b\in B$
  we have $g(b) = \sum_{}^{}{\lambda_ic_j}$. Hence $b - \sum_{}^{}{\lambda_ib_i}\in\ker g$. Since
  $\ker g= \im f$, $b - \sum_{}^{}{\lambda_ib_i} = f(a)$, so
  \begin{align*}
    b-\sum\lambda_jb_j=f\left(\sum\mu_ia_i\right)=\sum\mu_if(a_i).
  \end{align*}

  This shows that $b$ can be written as a linear combination of $\{b_j\}$ and $\{f(a_i)\}$. It
  is left to the reader to show that $\{b_j, f(a_i)\}$ are linearly independent.
\end{proof}

\begin{definition}\label{def:4-2}
  For a chain complex $A^* = \{\cdots\lr{} A^{p-1}\lr{\dd^{p-1}} A^p\lr{\dd^p} A^{p+1}\lr{}\cdots\}$, we define
  the $p$-th cohomology vector space to be
  \begin{align*}
    H^p(A^*) = \ker\dd^p/\im\dd^{p-1}.
  \end{align*}

  The elements of Ker $\dd^p$ are called $p$-cycles (or are said to be closed) and the
  elements of $\im\dd^{p-1}$ are called $p$-boundaries (or said to be exact). The elements
  of $H^p(A^*)$ are called \Index{cohomology classes}.

  A chain map $f:A^*\to B^*$ between chain complexes consists of a family $f^p: A^p\to B^p$ of linear maps, satisfying
  $\dd_B^p\circ f^p = f^{p+1}\circ\dd^p_A$. A chain map is illustrated as the commutative diagram


  \begin{center}
    \begin{tikzcd}
      \cdots \arrow[r] & A^{p-1} \arrow[r, "\dd^{p-1}"] \arrow[d, "f^{p-1}"]      & A^p \arrow[r, "\dd^p"] \arrow[d, "f^p"] & A^{p+1} \arrow[r] \arrow[d, "f^{p+1}"] & \cdots \\
      \cdots \arrow[r] & B^{p-1} \arrow[r, "\dd^{p-1}"] & B^p \arrow[r, "\dd^p"]  & B^{p+1} \arrow[r] & \cdots
    \end{tikzcd}
  \end{center}
\end{definition}

\begin{lemma}\label{lemma:4-3}
  A chain map $f:A^*\to B^*$ induces a linear map
  \begin{align*}
    f^* = H^*(f): H^p(A^*)\lr{} H^p(B^*), \text{ for all } p
  \end{align*}
\end{lemma}

\begin{proof}
  Let $\alpha\in A^p$ be a cycle ($\dd^pa = 0$) and $[a] = a + \im\dd^{p-l}$ its corresponding
  cohomology class in $H^p(A^*)$. We define $f^*([a]) = [f^p(a)]$. Two remarks are
  needed. First, we have $\dd^p_B f^p(a) = f^{p+1}\dd^p_A(a) = f^{p+1}(0) = 0$. Hence $f^p(a)$ is
  a cycle. Second, $[f^p(a)]$ is independent of which cycle $a$ we choose in the class
  $[a]$. If $[a_1] = [a_2]$ then $a_1-a_2\in\im\dd^{p-1}_A$, and $f^p(a_1-a_2) = f^p\dd^{p-1}_A(x)
    = \dd^{p-1}_Bf^{p-1}(x)$. Hence $f^p(a_1) - f^p(a_2) \in\im\dd^{p-1}_B$, and $f^p(a_1), f^p(a_2)$ define the
  same cohomology class.
\end{proof}

A category $C$ consists of ``objects'' and ``morphisms'' between them, such that
``composition'' is defined. If $f:C_1\to C_2$ and $g:C_2\to C_3$ are morphisms, then
there exists a morphism $g\circ f: C_1\to C3$. Furthermore it is to be assumed that
$\id_C: C\to C$ is a morphism for every object $C$ of $C$. The concept is best illustrated
by examples:

\begin{itemize}
  \item The category of open sets in Euclidean spaces, where the morphisms are
        the smooth maps.
  \item The category of vector spaces, where the morphisms are the linear maps.
  \item The category of abelian groups, where the morphisms are homomor phisms.
  \item The category of chain complexes, where the morphisms are the chain maps.
  \item A category with just one object is the same as a semigroup, namely the semigroup of morphisms of the object.
  \item Every partially ordered set is a category with one morphism from $c$ to $d$, when $c\le d$.
\end{itemize}

A contravariant functor $F:\C{C}\to \C{V}$ between two categories maps every object
$C\in\ob\C{C}$ to an object $F(C)\in\ob\C{V}$, and every morphism $f:C_1\to C_2$ in $\C{C}$ to
a morphism $F(f):F(C_2)\to F(C_1)$ in $\C{V}$, such that
\begin{align*}
  F(g\circ f)=F(f)\circ F(g),\quad F(\id_C)=\id_{F(C)}.
\end{align*}

A \Index{covariant functor} $F:\C{C}\to\C{V}$ is an assignment in which $F(f):F(C_1)\to F(C_2)$, and
\begin{align*}
  F(g\circ f)=F(g)\circ F(f),\quad F(\id_C)=\id_{F(C)}.
\end{align*}

Functors thus are the ``structure-preserving'' assignments between categories. The
contravariant ones change the direction of the arrows, the covariant ones preserve
directions. We give a few examples:

\begin{itemize}
  \item Let $A$ be a vector space and $F(C) = \hom(C, A)$, the linear maps from
        $C$ to $A$. For $\phi:C_1\to C_2$, $\hom(\phi, A):\hom(C_2, A)\to \hom(C_2,A)$ is
        given by $\hom(\phi, A)(\psi) = \psi\circ\phi$. This is a contravariant functor from the
        category of vector spaces to itself.
  \item $F(C) = \hom(C, A), F(\phi):\psi \to\phi\circ\psi$. This is a covariant functor from the
        category of vector spaces to itself.
  \item Let $\C{U}$ be the category of open sets in Euclidean spaces and smooth maps,
        and Vect the category of vector spaces. The vector space of differential
        $p$-forms on $U\in\C{U}$ defines a contravariant functor
        \begin{align*}
          \Omega^p(U):\C{U}\to \R{Vect}.
        \end{align*}
  \item Let Vect* be the category of chain complexes. The de Rham complex
        defines a contravariant functor $\Omega^*:\C{U}\to\R{Vect}^*$.
  \item For every $p$ the homology $H^p:\R{Vect}^*\to\R{Vect}$ is a covariant functor.
  \item The composition of the two functors above is exactly the de Rham
        cohomology functor $H^p:\C{U}\to\R{Vect}$. It is contravariant.
\end{itemize}

A short exact sequence of chain complexes
\begin{align*}
  0\lr{} A^*\lr{f} B^*\lr{g} C^*\lr{} 0
\end{align*}

consists of chain maps $f$ and $g$ such that $0\lr{}A^p\lr{f}B^p\lr{g}C^p\lr{}0$ is exact
for every $p$.

\begin{lemma}\label{lemma:4-4}
  For a short exact sequence of chain complexes the sequence
  \begin{align*}
    H^p(A^*)\lr{f^*} H^p(B^*)\lr{g^*} H^p(C^*)
  \end{align*}

  is exact.
\end{lemma}

\begin{proof}
  Since $g^p\circ f^p=0$, we have
  \begin{align*}
    g^*\circ f^*([ a ])=g^*([f^p(a)])=[g^p(f^p(a))]=0
  \end{align*}

  for every cohomology class $[a]\in H^p(A^*)$. Conversely, assume for $[b]\in H^p(B)$ that
  $g^*[b] = 0$. Then $g^p(b) = \dd^{p-1}_C(c)$. Since $g^{p-1}$ is surjective, there exists
  $b_1\in B^{p-1}$ with $g^{p-1}(b_1) = c$. It follows that $g^p(b-\dd^{p-1}_B(b_1)) = 0$. Hence
  there exists $a\in A^p$ with $f^p(a) = b-\dd^{p-1}_B(b_1)$. We will show that $a$ ia a $p$-cycle.
  Since $f^{p+1}$ is injective, it is sufficient to note that $f^{p+1}(\dd^p_A(a)) = 0$. But
  \begin{align*}
    f^{p+1}\big(d_A^p(a)\big)=d_B^p\big(f^p(a)\big)=d_B^p\big(b-d_B^{p-1}(b_1)\big) = 0
  \end{align*}

  since $b$ is a $p$-cycle and $\dd^p\circ\dd^{p-1} = 0$. We have thus found a cohomology class
  $[a]\in H^p(A)$, and $f^*([a]) = [b - \dd^{p-1}_B(b_1)]$.
\end{proof}


One might expect that the sequence of Lemma \ref{lemma:4-4} could be extended to a short
exact sequence, but this is not so. The problem is that, even though $g^p:B^p\to C^p$
is surjective, the pre-image $(g^P)^{-1}(c)$ of a $p$-cycle with $c\in C^p$ need not contain
a cycle. We shall measure when this is the case by introducing.


\begin{definition}
  For a short exact sequence of chain complexes $0\lr{}A^*\lr{f}B^*\lr{g}C^*\lr{}0$, we define
  \begin{align*}
    \partial^*: H^p(C^*) \to H^{p+1}(A^*)
  \end{align*}

  to be the linear map given by
  \begin{align*}
    \partial^*([c]) = \left[\left(f^{p+1}\right)^{-1} \left(\dd^p_B\left((g^p)^{-1}(c)\right)\right)\right]
  \end{align*}
\end{definition}

There are several things to be noted. The definition expresses that for every $b\in (g^p)^{-1}(c)$ we have
$\dd^p_B(b)\in\im(f^{p+1})$, and that the uniquely determined $a\in A^{p+1}$ with $f^{p+1}(a) = \dd^p_B(b)$ is a
$p+1$-cycle. Finally it is postulated that $[a]\in H^{p+1}(A^*)$ is independent of the choice of $b\in (g^b)^{-1}(c)$.

In order to prove these assertions it is convenient to write the given short exact
sequence in a diagram:


\begin{center}
\tikzcdset{
  arrow style=tikz,
  % diagrams={>=Latex}
}
\begin{tikzcd}
            & \arrow[d]                                               &   \arrow[d]                                               &   \arrow[d]                                       &   \\
0\arrow[r]  & A^{p-1} \arrow[r, "f^{p-1}"] \arrow[d, "\dd^{p-1}_A"]   &   B^{p-1} \arrow[r, "g^{p-1}"] \arrow[d, "\dd^{p-1}_B"]   &   C^{p-1} \arrow[r] \arrow[d, "\dd^{p-1}_C"]      & 0 \\
0\arrow[r]  & A^p \arrow[r, "f^p"] \arrow[d, "\dd^p_A"]               &   B^p \arrow[r, "g^p"] \arrow[d, "\dd^p_B"]               &   C^p \arrow[r] \arrow[d, "\dd^p_C"] \arrow[dll]  & 0 \\
0\arrow[r]  & A^{p+1} \arrow[r, "f^{p+1}"] \arrow[d]                  &   B^{p+1} \arrow[r, "g^{p+1}"] \arrow[d]                  &   C^{p+1} \arrow[r] \arrow[d]                     & 0 \\
\mbox{}     &  \mbox{}                                                &   \mbox{}                                                 &   \mbox{}                                          & \mbox{}  \\
\end{tikzcd}
\end{center}

The slanted arrow indicates the definition of $\partial^*$. We shall now prove the necessary
assertions which, when combined, make $\partial^*$ well-defined. Namely:
\begin{enumerate}[label=(\roman*)]
  \item If $g^p(b) = c$  and $\dd^p_C(c) = 0$, then $\dd^p_B(b) \in f^{p+1}$.
  \item If $f^{p+1}(a) = \dd^p_B(b)$, then $\dd^{p+1}_A(a) = 0$.
  \item If $g^p(b_1)  g^p(b) = c$ and $f^{p+1}(a_i) = \dd^p_B(b_i)$, then $[a_1]=[a_2] \in H^{p+1}(A^*)$.
\end{enumerate}