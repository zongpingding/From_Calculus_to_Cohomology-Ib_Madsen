\chapter*{preface}
\phantomsection
\addcontentsline{toc}{chapter}{Preface}

This text offers a self-contained exposition of the cohomology of differential
forms, de Rham cohomology, and of its application to characteristic classes
defined in terms of the curvature tensor. The only formal prerequisites are knowledge 
of standard calculus and linear algebra, but for the later part of the
book some prior knowledge of the geometry of surfaces, Gaussian curvature, will
not hurt the reader.

The first seven chapters present the cohomology of open sets in Euclidean spaces
and give the standard applications usually covered in a first course in algebraic
topology, such as Brouwer's fixed point theorem, the topological invariance of
domains and the Jordan-Brouwer separation theorem. The next four chapters
extend the definition of cohomology to smooth manifolds, present Stokes' theorem 
and give a treatment of degree and index of vector fields, from both the cohomological 
and geometric point of view. 

The last ten chapters give the more advanced part of cohomology:
the Poincar\'{e}-Hopf theorem, Poincare duality, Chern classes, the Euler class, and
finally the general Gauss-Bonnet formula. As a novel point we prove the so
called splitting principles for both complex and real oriented vector bundles.
The text grew out of numerous versions of lecture notes for the beginning course
in topology at Aarhus University. The inspiration to use de Rham cohomology as
a first introduction to topology comes in part from a course given by G. Segal at
Oxford many years ago, and the first few chapters owe a lot to his presentation
of the subject. It is our hope that the text can also serve as an introduction to the
modern theory of smooth four-manifolds and gauge theory.

The text has been used for third and fourth year students with no prior exposure
to the concepts of homology or algebraic topology. We have striven to present
all arguments and constructions in detail. Finally we sincerely thank the many
students who have been subjected to earlier versions of this book. Their comments
have substantially changed the presentation in many places.

\vspace*{2em}
\noindent{Aarhus, January 1996}
\newpage