\chapter{The Alternating Algebra}
Let $V$ be a vector space over $\RR$. A map
\begin{align*}
  f: \underbrace{V\times V\times \cdots \times V}_{k\text{ times}} \to \RR
\end{align*}

is called \Index{$k$-linear} (or \Index{multilinear}), if $f$ is linear in each factor.


\begin{definition}\index{alternating!map}
  A $k$-linear map $\omega: V^k\to\RR$ is said to be alternating if
  $\omega(\xi_1, \cdots, \xi_k) = 0$ whenever $\xi_i=\xi_j$ for some pair $i\neq j$. The vector space
  of alternating, $k$-linear maps is denoted by $\alt^k(V)$\index{$\alt^k(V)$}.
\end{definition}


We immediately note that $\alt^k(V) = 0$ if $k > \dim V$. Indeed, let $e_1, \cdots, e_n$ be a
basis of $V$, and let $\omega\in\alt^k(V)$. Using multilinearity,
\begin{align*}
  \omega(\xi_{1},\ldots,\xi_{k})=\omega\Big(\sum\lambda_{i,1}e_{i},\ldots,\sum\lambda_{i,k}e_{i}\Big)=\sum\lambda_{J}\omega(e_{j_{1}},\ldots,e_{j_{k}})
\end{align*}


with $\lambda_J = \lambda_{j_1, 1}, \cdots, \lambda_{j_k, k}$. Since $k>n$, there must be at least one repetition
among the elments $e_{j_1}, \cdots, e_{j_k}$. Hence $\omega(e_{j_1}, \cdots, e_{j_k}) = 0$.

The symmetric group of permutations of the set $\{1, \cdots,k\}$ is denoted by $S(k)$.
We remind the reader that any permutation can be written as a composition of
transpositions. The transposition that interchanges $i$ and $j$ will be denoted by $(i, j)$.
Furthemiore, and this fact will be used below, any permutation can be written as a
composition of transpositions of the type $(i, i+1), (i, i+1)\circ(i+1, i+2)\circ(i, i+1) =
  (i, i + 2)$ and so forth. The sign of a permutation:
\begin{align}\label{eq:2-1}
  \sign: S(k) \to \{\pm 1\}
\end{align}


is a homomorphism, $\sign(\sigma\circ\tau) = \sign(\sigma)\circ\sign(\tau)$, which maps every
transposition to $-1$. Thus the sign of $\sigma\in S(k)$ is $-1$ precisely if $\sigma$ decomposes into a
product consisting of an odd number of transpositions.

\begin{lemma}
  If $\omega\in\alt^k(V)$ and $\sigma\in S(k)$, then
  \begin{align*}
    \omega\big(\xi_{\sigma(1)},\ldots,\xi_{\sigma(k)}\big)=\mathrm{sign}(\sigma)\omega(\xi_{1},\ldots,\xi_{k}).
  \end{align*}
\end{lemma}

\begin{proof}
  It is sufficient to prove the formula when $\sigma = (i, j)$. Let
  \begin{align*}
    \omega_{i,j}(\xi,\xi')=\omega(\xi_{1},\ldots,\xi,\ldots,\xi',\ldots,\xi_{k}),
  \end{align*}

  with $\xi$ and $\xi'$ eoccurring at positions $i$ and $j$ respectively. The remaining $\xi_p\in V$
  are arbitrary but fixed vectors. From the definition it follows that $\omega_{i,j}\in\alt^2(V)$.
  Hence $\omega_{i,j} (\xi_i + \xi_j, \xi_i + \xi_j) = 0$. Bilinearity yields that
  $\omega_{i,j} (\xi_i + \xi_j) + \omega_{i,j} (\xi_j + \xi_i) = 0$
\end{proof}


\begin{example}\label{example:2-3}
  Let $V = \RR^k$ and $\xi_i = (\xi_{i1}, \cdots, \xi_{ik})$. The function $\omega(\xi_1, \cdots, \xi_k) = \det((\xi_{ij}))$
  is alternating, by the calculational rules for determinants.
\end{example}

We want to define the \Index{exterior product}
\begin{align*}
  \wedge: \alt^p(V)\times\alt^q(V) \to \alt^{p+q}(V).
\end{align*}

When $p=q=1$, it is given by $(\omega_1\wedge\omega_2) = \omega_1(\xi_1)\omega_2(\xi_2) - \omega_2(\xi_1)\omega_1(\xi_2)$.

\begin{definition}\label{def:2-4}
  A \Index{$(p, q)$-shuffle} $\sigma$ is a permutation of the set $\{1, \cdots, p + q\}$ satisfying
  \begin{align*}
    \sigma(1) < \cdots < \sigma(p) \quad\text{and}\quad \sigma(p + 1) < \cdots < \sigma(p + q).
  \end{align*}

  The set of all such permutations is denoted by $S(p, q)$. Since a $(p, q)$-shuffle is uniquely determined by the
  set $\{\sigma(1),\cdots,\sigma(p)\}$, the cardinality of $S(p, q)$ is $\binom{p + q}{p}$.
\end{definition}


\begin{definition}[Exterior product]\label{def:2-5}
  For $\omega_1\in\alt^p(V)$ and $\omega_2\in\alt^q(V)$, we defined
  \begin{align*}
    (\omega_1\wedge\omega_2) & (\xi_1,\ldots,\xi_{p+q})                                                             \\
    =                        & \sum_{\sigma\in S(p,q)}\sign(\sigma)\omega_1(\xi_{\sigma(1)},\ldots,\xi_{\sigma(p)})
    \cdot\omega_2(\xi_{\sigma(p+1)},\ldots,\xi_{\sigma(p+q)}).
  \end{align*}

  It is obvious that $\omega_1\wedge\omega_2$ is a $(p+q)$-linear map, but moreover.
\end{definition}

\begin{lemma}\label{lemma:2-2}
  If $\omega_1\in\alt^p(V)$ and $\omega_2\alt^q(V)$ then $\omega_1\wedge\omega_2\in\alt^{p+q}(V)$.
\end{lemma}

\begin{proof}
  We first show that $\omega_1\wedge\omega_2(\xi_1,, \xi_2, \cdots, \xi_{p+q}) = 0$ when $\xi_i = \xi_j$.
  We let
  \begin{enumerate}[label=(\roman*)]
    \item $S_{12} = \{\sigma\in S(p, q) | \sigma(1) = 1, \sigma(p+1) = 2\}$
    \item $S_{21} = \{\sigma\in S(p, q) | \sigma(1) = 2, \sigma(p+1) = 1\}$
    \item $S_0 = S(p, q) - (S_{12} - S_{21})$
  \end{enumerate}

  If $\sigma\in S_0$ then either $\omega_1(\xi_{\sigma(1)}, \cdots, \xi_{\sigma(p)}) = 0$
  or $\omega_2(\xi_{\sigma(p+1)}, \cdots, \xi_{\sigma(p+q)}) = 0$, since $\xi_P\sigma(1) = \xi_{\sigma(2)}$
  or $\xi_{\sigma(p+1)} = \xi_{\sigma(p+2)}$. Left composition with  the transposition $\tau = (1, 2)$ is a bijecrtion
  $S_{12}\to S_{21}$. We therefore have
  \begin{align*}
      & (\omega_{1}\wedge\omega_{2})(\xi_{1},\xi_{2},\ldots,\xi_{p+q})                                                                                             \\
    = & \sum_{\sigma\in S_{12}}\sign(\sigma)\omega_1(\xi_{\sigma(1)},\ldots,\xi_{\sigma(p)})\omega_2(\xi_{\sigma(p+1)},\ldots,\xi_{\sigma(p+q)})                   \\
    - & \sum_{\sigma\in S_{12}}\sign(\sigma)\omega_1(\xi_{r\sigma(1)},\ldots,\xi_{\tau\sigma(p)})\cdot\omega_2(\xi_{\tau\sigma(p+1)},\ldots\xi_{\tau\sigma(p+q)}).
  \end{align*}

  Since $\sigma(1) = 1$ and $\sigma(p+1) = 2$, while $\tau\sigma(1) = 2$ and $\tau\sigma(p+1) = 1$, we see that
  $\tau\sigma(i) = \sigma(i)$ where $i\neq 1, p+1$. But $\xi_1 = \xi_2$ so the terms in the two sums cancel.
  The case $\xi_i = \xi_{i+1}$ is similar. Now $\omega_1\wedge\omega_2$ is alternating according to Lemma \ref{lemma:2-7} below.
\end{proof}

\begin{lemma}\label{lemma:2-7}
  A $k$-linear map $\omega$ is alternating if $\omega(\xi_1, \cdots, \xi_k) = 0$ for all $k$-tuples
  with $\xi_i = \xi_{i+1}$ for some $1\le i\le k-1$.
\end{lemma}


\begin{proof}
  $S(k)$ is generated by the transpositions $(i, i + 1)$, and by the argument
  of Lemma \ref{lemma:2-2},
  \begin{align*}
    \omega(\xi_1, \cdots, \xi_i, \xi_{i+1}, \cdots, \xi_k)
    = - \omega(\xi_1, \cdots, \xi_{i+1}, \xi_i, \cdots, \xi_k).
  \end{align*}

  Hence Lemma \ref{lemma:2-2} holds for all $\sigma\in S(k)$, and $\omega$ is alternating.
\end{proof}

It is clear from the definition that
\begin{align*}
  (\omega_{1}+\omega_{1}^{\prime})\wedge\omega_{2}
  = \omega_{1}\wedge\omega_{2}+\omega_{1}^{\prime}\wedge w_{2}            \\
  (\lambda\omega_{1})\wedge\omega_{2}
  = \lambda(\omega_{1}\wedge\omega_{2})=\omega_{1}\wedge\lambda\omega_{2} \\
  \omega_{1}\wedge\left(\omega_{2}+\omega_{2}^{\prime}\right)
  = \omega_{1}\wedge\omega_{2}+\omega_{1}\wedge\omega_{2}^{\prime}
\end{align*}

for $\omega_1, \omega_1' \in\alt^p(V)$ and $\omega_2, \omega_2'\in\alt^q(V)$.


\begin{lemma}\label{lemma:2-8}
  If $\omega_1\in\alt^p(V)$ and $\omega_2\in\alt^q(V)$, then $\omega_1\wedge\omega_2 = (-1)^{pq}\omega_2\wedge\omega_1$.
\end{lemma}


\begin{proof}
  Let $\tau\in S(p+q)$ be the element with
  \begin{align*}
    \begin{array}{cccc}
      \tau(1)  = p+1 , & \tau(2)  = p+2, & \cdots, & \tau(q) = p+q. \\
      \tau(q+1) = 1,   & \tau(q+2) = 2,  & \cdots, & \tau(p+q) = p.
    \end{array}
  \end{align*}

  We have $\sign(\tau) = (-1)^{pq}$. Composition with $\tau$ defines bijiection
  \begin{align*}
    S(p, q) \lr{\cong} S(q, p), \quad \sigma\mapsto\tau\circ\sigma
  \end{align*}

  Note that
  \begin{align*}
    \omega_2(\xi_{\sigma\gamma(1)}, \cdots, \xi_{\sigma\gamma(q)})
    & = \omega_2(\xi_{\tau\sigma(p+1)}, \cdots, \xi_{\tau\sigma(p+q)}). \\
    \omega_1(\xi_{\sigma\gamma(q+1)}, \cdots, \xi_{\sigma\gamma(p+q)})
    & = \omega_1(\xi_{\tau\sigma(1)}, \cdots, \xi_{\tau\sigma(p)}).
  \end{align*}

  Hence
  \begin{align*}
    & \omega_{2}\wedge\omega_{1}(\xi_{1},\ldots,\xi_{p+q})                                                         \\
    & = \sum_{\sigma\in S(q,p)}\sign(\sigma)\omega_{2}\big(\xi_{\sigma(1)},\ldots,\xi_{\sigma(q)}\big)
  \omega_{1}\big(\xi_{\sigma(q+1)},\ldots,\xi_{\sigma(p+q)}\big)                                                  \\
    & = \sum_{\sigma\in S(p,q)}\sign(\sigma\tau)\omega_{2}\big(\xi_{\sigma\tau(1)},\ldots,\xi_{\sigma\tau(q)}\big)
  \omega_{1}\big(\xi_{\sigma\tau(q+1)},\ldots,\xi_{\sigma\tau(p+q)}\big)                                          \\
    & = (-1)^{pq}\sum_{\sigma\in S(p,q)}\sign(\sigma)\omega_{1}\big(\xi_{\sigma(1)},\ldots,\xi_{\sigma(p)}\big)
  \omega_{2}\big(\xi_{\sigma(p+1)},\ldots,\xi_{\sigma(p+q)}\big)                                                  \\
    & = (-1)^{pq}\omega_{1}\wedge\omega_{2}(\xi_{1},\ldots,\xi_{p+q}).
  \end{align*}
\end{proof}


\begin{lemma}
  If $\omega_1\in\alt^p(V)$ and $\omega_2\in\alt^q(V)$ and $\omega_3\in\alt^r(V)$, then
  \begin{align*}
    \omega_1\wedge(\omega_2\wedge\omega_3) = (\omega_1\wedge\omega_2)\wedge\omega_3.
  \end{align*}
\end{lemma}

\begin{proof}
  Let $S(p, q, r)\subseteq S(p+q+r)$ consist of the permutations $\sigma$ with
  \begin{align*}
    \sigma(1) <     & \cdot < \sigma(p)      \\
    \sigma(p+1) <   & \cdot < \sigma(p+q)    \\
    \sigma(p+q+1) < & \cdot < \sigma(p+q+r).
  \end{align*}

  We will need the subset $S(\tilde{p}, q, r)$ and $S(p, q, \sim{r})$ of $S(p, q, r)$ given by
  \begin{align*}
  \sigma \in S(\tilde{p}, q, r) \Longleftrightarrow \sigma \text{ is the identity on }  & \{1, \cdots, p\} \text{ and } \sigma\in S(p, q, r) \\
  \sigma \in S(p, q, \tilde{r}) \Longleftrightarrow \sigma \text{ is the identity on }  & \{p+q+1, \cdots, p+q+r\}                           \\
                                                                                        & \text{ and } \sigma\in S(p, q, r)                  \\
  \end{align*}

  There are bijections
  \begin{align}\label{eq:2-2}
    \begin{aligned}
      S(p, q, r) \times S(p, q, r)         & \lr{\cong} S(p, q, r); \quad (\sigma, \tau)\mapsto\sigma\circ\tau   \\
      S(p, q, r) \times S(p, q, \tilde{r}) & \lr{\cong} S(p, q, r); \quad (\sigma, \tau)\mapsto\tau\circ\sigma.
    \end{aligned}
  \end{align}

  With these notations we have
  \begin{align*}
    & [\omega_1\wedge(\omega_2\wedge\omega_3)](\xi_1,\ldots,\xi_{p+q+r})                                                                              \\
    & = \sum_{\sigma\in S(p,q+r)}\sign(\sigma)\omega_1(\xi_{\sigma(1)},\ldots,\xi_{\sigma(p)})
    (\omega_2\wedge\omega_3)(\xi_{\sigma(p+1)},\ldots,\xi_{\sigma(p+q+r)})                                                                             \\
    & = \sum_{\sigma\in S(p,q+r)}\sign(\sigma)\sum_{\sigma\in S(p,q,r)}\sign(\tau)\Big[\omega_1(\xi_{\sigma(1)},\ldots,\xi_{\sigma(p)})               \\
    & \hspace*{2em} \omega_2(\xi_{\sigma\tau(p+1)},\ldots,\xi_{\sigma\tau(p+q)})\omega_3(\xi_{\sigma\tau(p+q+1)},\ldots,\xi_{\sigma\tau(p+q+r)})\Big] \\
    & = \sum_{u\in S(p,q,r)}\Big[\sign(u)\omega_1(\xi_{u(1)},\ldots,\xi_{u(p)})\omega_2(\xi_{u(p+1)},\ldots,\xi_{u(p+q)})                             \\
    & \hspace*{2em} \omega_3(\xi_{u(p+q+1)},\ldots,\xi_{u(p+q+r)})\Big]
  \end{align*}

  where the last equality follows from the first equation in \eqref{eq:2-2}. Quite analogously one can calculate
  $[(\omega_1\wedge\omega_2)\wedge\omega_3](\xi_1, \cdots, \xi_{p+q+r})$, employing the second equation in \eqref{eq:2-2}.
\end{proof}

\begin{remark}
In other textbook on alternating functions one can often see the definition
\begin{align*}
    & \omega_{1}\bar{\wedge}\omega_{2}(\xi_{1},\ldots,\xi_{p+q})\\
  = & \frac{1}{p!q!}\sum_{\sigma\in S(p+q)}\sign(\sigma)
      \omega_{1}(\xi_{\sigma(1)},\ldots,\xi_{\sigma(p)})
      \omega_{2}(\xi_{\sigma(p+1)},\ldots,\xi_{\sigma(p+q)}).
\end{align*}

Note that in this formula $\{\sigma(1), \cdots,\sigma(p)\}$ and $\{\sigma(p + 1), \cdots, \sigma(p + q)\}$ are not
ordered. There are exactly $S(p)\times S(q)$ ways to come from an ordered set to the arbitrary sequence above; 
this causes the factor $\frac{1}{p!q!}$, so $\omega_1\bar{\wedge}\omega_2 = \omega_1\wedge\omega_2$.
\end{remark}

An $\RR$-algebra $A$ consists of a vector space over $\RR$ and a bilinear map $\mu:A\times A\to A$
which is associative, $\mu(a, \mu(b,c)) = \mu(\mu(a, b), c)$ for every $a,b,c \in A$. The
algebra is called \Index{unitary}\index{unitary algebra} if there exists a unit element for $\mu, \mu(1, a) = \mu( a, 1) = a$
for all $a\in A$.

\begin{definition}\label{def:2-11}\;\par
  \begin{enumerate}[label=(\roman*)]
    \item A graded $\RR$-algebra $A_*$ is a sequence of vector spaces $A_k, k = 0,1, \cdots$,
      and bilinear maps $\mu:A_k\times A_l\to A_{k+l}$ which are associative.
    \item The algebra $A_*$ is called connected if there exists a unit element $1\in A_0$ and 
      if $\epsilon:\RR\to A_0$, given by $\epsilon(r) = r\cdot 1$, is an isomorphism.
    \item The algebra called (graded) commutative (or anti-commutative), if $\mu(a, b) = (-1)^{kl}\mu(b, a)$
      for all $a\in A_k$ and $b\in A_l$.
  \end{enumerate}
\end{definition}


The elements in $A_k$ are said to have degree $k$. The set $\alt^k(V)$ is a vector space
over $\RR$ in the usual manner:
\begin{align*}
  (\omega_1+\omega_2)(\xi_1,\dots,\xi_k) & = \omega_1(\xi_1,\dots,\xi_k)+\omega_2(\xi_1,\dots,\xi_k)\\
  (\lambda\omega)(\xi_1,\dots,\xi_k)     & = \lambda\omega(\xi_1,\dots,\xi_k),\quad\lambda\in\mathbb{R}. 
\end{align*}


The product from Definition \ref{def:2-5} is a bilinear map from $\alt^p(V)\times\alt^q(V)$ to 
$\alt^{p+q}(V)$. We set $\alt^0(V)=\RR$ and expand the product to $\alt^0(V)\times\alt^p(V)$ by
using the vector space structure. The basic formal properties of the alternating forms can now be 
summarized in. 


\begin{theorem}\label{theorem:2-12}
  $Alt^*(V)$ is an anti-commutative and connected graded algebra.
\end{theorem}

$Alt^*(V)$ is called the exterior\index{exterior algebra} or alternating algebra\index{alternating!algebra} associated to $V$.


\begin{lemma}\label{lemma:2-13}
  For 1-forms $\omega_1, \cdots, \omega_p\in\alt^1(V)$, we have 
  \begin{align*}
    (\omega_1\wedge\dots\wedge\omega_p)(\xi_1,\dots,\xi_p)
    = \det
    \begin{pmatrix}
      \omega_1(\xi_1) & \omega_1(\xi_2) & \cdots & \omega_1(\xi_p)\\
      \omega_2(\xi_1) & \omega_2(\xi_2) & \cdots & \omega_2(\xi_p)\\
      \vdots          & \vdots          &        & \vdots\\
      \omega_p(\xi_1) & \omega_p(\xi_2) & \cdots & \omega_p(\xi_p)
    \end{pmatrix}
  \end{align*}
\end{lemma}

\begin{proof}
  The case $p=2$ is abvious. We procced by induction on $p$. According to Definition \ref{def:2-5},
  \begin{align*}
      & \omega_{1}\wedge(\omega_{2}\wedge\ldots\wedge\omega_{p})(\xi_{1},\ldots,\xi_{p})\\
    = & \sum_{j=1}^{p}(-1)^{j+1}\omega_{1}(\xi_{j})(\omega_{2}\wedge\ldots\wedge\omega_{p})
      \Big(\xi_{1},\quad,\dot{\xi}_{j},\ldots,\xi_{p}\Big)
  \end{align*}

  where $(\xi_1, \cdots, \hat{\xi_j}, \cdots, \xi_p)$ denotes the $p-1$-tuple where 
  $\xi_j$ has been omitted. The lemma follows by expanding the determinant by the first row.
\end{proof}


Note, from Lemma \ref{lemma:2-13}, that if the 1-forms $\omega_1, \cdot, \omega_p\in\alt^1(V)$ 
are linearly independent then $\omega_1\wedge\cdots\wedge\omega_p\neq 0$. Indeed, we can choose 
elements $\xi\in V$ with $\omega_i(\xi_i) = 0$ for $i\neq j$ and $\omega_j(\xi_j) = 0$, so that 
$\det(\omega_i(\xi_j)) = 1$. Conversely, if $\omega_1, \cdots, \omega_p$ are linearly dependent,
we can express one of them, say $\omega_p$, as a linear combination of the others. If 
$\omega_p = \sum_{i=1}^{p-1}{r_i\omega_i}$, then
\begin{align*}
  \omega_{1}\wedge\cdots\wedge\omega_{p-1}\wedge\omega_{p}
  = \sum_{i=1}^{p-1}r_{i}\omega_{1}\wedge\cdots\wedge\omega_{p-1}\wedge\omega_{i}=0,
\end{align*}

as the determinant in Lemma \ref{lemma:2-13} has two rows. We have proved.



\begin{lemma}
  For 1-form $\omega_1, \cdots, \omega_p$ on $V$, $\omega_1\wedge\cdots\wedge\omega_p \neq 0$ if and only if
  they are linearly independent.
\end{lemma}

\begin{theorem}\label{theorem:2-15}
Let $e_1, \cdots, e_n$ be a basis of $V$ and $\epsilon_1, \cdots, \epsilon_n$ the dual basis of $V^*$. Then 
\begin{align*}
  \left\{\epsilon_{\sigma(1)}\wedge\cdots\wedge\epsilon_{\sigma(k)}\right\}_{\sigma\in S(p, n-p)}
\end{align*}

is a basis of $\alt^p(V^*)$. In particular
\begin{align*}
  \dim\alt^p(V^*) = \binom{\dim V}{p}.
\end{align*}
\end{theorem}

\begin{proof}
  Since $\epsilon_i{e_j} = 0$ when $i\neq j$, and $\epsilon_i{e_j}= 1$, Lemma \ref{lemma:2-13} gives 
  \begin{align}\label{eq:2-3}
    \epsilon_{i_1}\wedge\cdots\wedge\epsilon_{i_p}(e_{j_1},\ldots,e_{j_p})
    = \left\{\begin{aligned}
      & 0             && \text{ if } \{i_1, \cdots, i_p\}\neq \{j_1, \cdots, j_p\},\\
      & \sign(\sigma) && \text{ if } \{i_1, \cdots, i_p\} = \{j_{\sigma(1)}, \cdots, j_{\sigma(p)}\}.
    \end{aligned}\right.
  \end{align}

  Here $\sigma$ is the permutation $\epsilon(i_k) = j_k$. From Lemma \ref{lemma:2-13} and \eqref{eq:2-3} we get
  \begin{align*}
    \omega = \sum_{\sigma\in S(p,n-p)}\omega\big(e_{\sigma(1)},\ldots,e_{\sigma(p)}\big)
      \epsilon_{\sigma(1)}\wedge\ldots\wedge\epsilon_{\sigma(p)}
  \end{align*}

  for any alternating $p$-form. Thus $\epsilon_{\sigma(1)}\wedge\cdots\wedge\epsilon_{\sigma(p)}$ generates 
  the vector space $\alt^p(V)$. Linear independence follows from \eqref{eq:2-3}, since a relation
  \begin{align*}
    \sum_{\sigma\in S(p,n-p)}\lambda_{\sigma}\epsilon_{\sigma(1)}\wedge\ldots\wedge\epsilon_{\sigma(p)}=0,\quad\lambda_{\sigma}\in\mathbb{R}
  \end{align*}

  evaluated on $(e_{\sigma(1)}, \cdots, e_{\sigma(p)})$ gives $\lambda_{\sigma} = 0$.
\end{proof}

Note from Theorem 2.15 that $\alt^n(V)\lr{\cong}\RR$ if $n = \dim V$ and, as mentioned
earlier, that $\alt^p(V) = 0$ if $p > n$. A basis of $\alt^n(V)$ is given by $\epsilon_{1}\wedge\cdots\wedge\epsilon_{n}$.
In particular every alternating $n$-fonn on $\RR^n$ is proportional to the form in Example \ref{example:2-3}.

A linear map $f:V\to W$ induces the linear map 
\begin{align}\label{eq:2-4}
  \alt^p(f): \alt^p(W)\to\alt^p(V)
\end{align}

by setting $\alt^p(f)(W)(\xi_1, \cdots, \xi_p) = \omega(f(\xi_1), \cdots, f(\xi_p))$. For the composition of
maps we have $\alt^p(g\circ f) = \alt^p(f)\circ\alt^p(g)$, and $\alt^P(\id) = \id$. These two
properties are summarized by saying that $\alt^p(-)$ is a \Index{contravariant functor}\index{{functor!contravariant}}. If $\dim V = n$ and 
$f:V\to V$ is a linear map then
\begin{align*}
  \alt^p(f): \alt^n(V)\to\alt^n(V)
\end{align*}

is a linear endomorphism of 1-dimensional vector space and thus multiplication by a number $d$. From Theorem 
\ref{theorem:2-16} below it follows that $d = \det(f)$. We shall also be using other maps 
\begin{align*}
  \alt^p(f): \alt^p(V)\to\alt^p(V)
\end{align*}

Let $\tr(g)$ denotes the trace of a linear endomorphism $g$.


\begin{theorem}\label{theorem:2-16}
  The characteristic polynomial ofa linear endomorphism $f:V\to V$ is given by
  \begin{align*}
    \det(f-t)=\sum\limits_{i=0}^n{(-1)}^i\mathrm{tr}\Big(\mathrm{Alt}^{n-i}(f)\Big)t^i,
  \end{align*}

  when $n = \dim V$.
\end{theorem}

\begin{proof}
Choose a basis $e_1, \cdots, e_n$ of $V$ Assume first that $e_1, \cdots, e_n$ are eigenvectors of $f$,
\begin{align*}
  f(e_i) = \lambda_i e_i, i = 1, \cdots, n.
\end{align*}

Let $\epsilon_1, \cdots, \epsilon_n$ be the dual basis of $\alt^1(V)$. Then
\begin{align*}
  \mathrm{Alt}^p(f)\big(\epsilon_{\sigma(1)}\wedge\cdots\wedge\epsilon_{\sigma(p)}\big)
  = \lambda_{\sigma(1)}\cdot\cdots\lambda_{\sigma(p)}\epsilon_{\sigma(1)}\wedge\cdots\wedge\epsilon_{\sigma(p)}
\end{align*}

and 
\begin{align*}
  \tr \alt^p(f) = \sum_{\sigma\in S(p, n-p)}^{}{\lambda_{\sigma(1)}\cdot\cdots\lambda_{\sigma(p)}}.
\end{align*}

On the other hand
\begin{align*}
  \det(f-t) 
  = \prod_{i=1}^{n}{(\lambda_i-t)} 
  = \sum{(-1)^{n-p}\left(\sum{\lambda_{\sigma(1)}\cdot\cdots\lambda_{\sigma(p)}}\right)t^{n-p}}.
\end{align*}

This proves the formula when $f$ is diagonal.

If $f$ is replaced by $gfg^{-l}$, with $g$ an isomorphism on $V$, then both sides of the
equation of Theorem \ref{theorem:2-16} remain unchanged. This is obvious for the left-hand
side and follows for the right-hand side since
\begin{align*}
  \alt^p(gfg^{-1}) = \alt^p(g)^{-1}\circ\alt^p(f)\circ\alt^p(g).
\end{align*}

by the functor property. Hence $\tr\alt^p(g\circ f\circ g^{-l}) = \tr\alt^p(f)$. Consider the
set
\begin{align*}
  D = \{gf^{-1}g^{-1} | f \text{ diagonal }, g \in GL(V)\}.
\end{align*}

If $V$ is a vector space over $\B{C}$ and all maps are complex linear, then $D$ is dense
in the set of linear endomorphisms on $V$. We shall not give a formal proof of
this, but it follows since every matrix with complex entries can be approximated
arbitrarily closely by a matrix for which all roots of the characteristic polynomial
are distinct. Since eigenvectors belonging to different eigenvalues are linearly
independent, $V$ has a basis consisting of eigenvectors for such a matrix, which
then belongs to $D$.

For general $f\in\R{End}(V)$ we can choose a sequence $d_n\in D$ with $d_n\to f$ 
(i.e. the $(i, j)$-th element in dn converges to the $(i, j)$-th element in $f$). 
Since both sides in the equation we want to prove are continuous, and since the 
equation holds for $d_n$ , it follows for $f$.
\end{proof}


It is not true that the set of diagonalizable matrices over $\RR$ is dense in the set of
matrices over $\RR - a$ matrix with imaginary eigenvalues cannot be approximated
by a matrix of the form $gfg^{-1}$, with f a real diagonal matrix. Therefore in the
proof of Theorem \ref{theorem:2-16} we. must pass to complex linear maps, even if we are
mainly interested in real ones.