\chapter{Introduction}
It is well-known that a continuous real function, that is defined on an open set of
\RR has a primitive function. How about multivariable functions? For the sake of
simplicity we restrict ourselves to smooth (or $C^\infty$-) functions, i.e. functions that
have continuous partial derivatives of all orders.
We begin with functions of two variables. Let $f:U\to \RR^2$ be a smooth function
.defined on an open set of $\RR^2$ .

\begin{question}\label{question:1-1}
Is there a smooth function $F:U\to \RR$, such that:
\begin{align}
  \frac{\partial F }{\partial x_1 } = f_1 \text{ and }
  \frac{\partial F }{\partial x_2 } = f_2, \text{ where } 
  f = (f_1, f_2)?
  \label{eq:1-1}
\end{align}

Since
\begin{align*}
  \frac{\partial ^2 F }{\partial x_2\partial x_1 } = 
  \frac{\partial ^2 F }{\partial x_1\partial x_2 }
\end{align*}

we must have 
\begin{align}
  \frac{\partial f_1 }{\partial x_2 } = \frac{\partial f_2 }{\partial x_1}
  \label{eq:1-2}
\end{align}

The correct question is therefore whether $F$ exists, assuming $f = (f_1, f_2)$ satisfies
\eqref{eq:1-2}. Is condition \eqref{eq:1-2} also sufficient?
\end{question}


\begin{example}\label{example:1-2}
Consider the function $f:\RR^2 \to \RR^2$ given by
\begin{align*}
  f(x_1, x_2) = \left(\frac{-x_2}{x_1^2 +x_2^2}, \frac{x_1}{x_1^2 +x_2^2},\right)
\end{align*}
\end{example}

It is easy to show that \eqref{eq:1-2} is satisfied. However, there is no function $F:\RR^2-\{0\}\to \RR$
that satisfies \eqref{eq:1-1}. Assume there were; then

\begin{align*}
  \int_{0}^{2\pi}{\frac{\dd }{\dd\theta} F(\cos\theta, \sin\theta)\;\mathrm{d}\theta}
  = F(1, 0) - F(1, 0) = 0
\end{align*}

On the other hand the chain rule gives
\begin{align*}
  \frac{\dd }{\dd\theta} F(\cos\theta, \sin\theta) 
  & = \frac{\dd F}{\dd x}\cdot (-\sin\theta) + \frac{\dd F}{\dd y}\cdot \cos\theta \\
  & = -f_1 (\cos\theta, \sin\theta)\cdot\sin\theta + f_2(\cos\theta, \sin\theta)\cdot \cos\theta\\
  & = 1
\end{align*}

This contradiction can only be explained by the non-existence of $F$.

\begin{definition}
  A subset $X\subset \RR^n$ is said to be star-shaped with respect to the point $x_0\in X$ if
  the line segemtn $\{tx_0+(1-t)x|t\in[0,1]\}$ is contained in $X$ for all $x\in X$.
\end{definition}

\begin{theorem}\label{theorem:1-4}
  Let $U\subset \RR^2$ be star-shaped. Then for any smooth function $f:U\to \RR^2$ that satisfies
  \eqref{eq:1-2}, Question \ref{question:1-1} has a solution.
\end{theorem}

\begin{proof}
  For the sake of simplicity we assume that $x_0=0\in\RR^2$. Consider the function $F:U\to \RR$.
  \begin{align*}
    F(x_1, x_2) = \int_{0}^{1}{\left[ x_1f_1(tx_1, tx_2) + x_2f_2(tx_1, tx_2) \right] \;\mathrm{d}t}.
  \end{align*}
Then one has

\begin{align*}
  \frac{\partial F}{\partial x_1} 
  = \int_{0}^{1}{\left[ f_1(tx_1, tx_2) + tx_1\frac{\partial f_1}{\partial x_1}(tx_1, tx_2) + tx_2\frac{\partial f_2}{\partial x_1}(tx_1, tx_2) \right] \;\mathrm{d}t}
\end{align*}

and 
\begin{align*}
  \frac{\dd }{\dd t} tf_1(tx_1, tx_2) 
  = f_1(tx_1, tx_2) + tx_1\frac{\partial f_1}{\partial x_1}(tx_1, tx_2) + tx_2\frac{\partial f_2}{\partial x_1}(tx_1, tx_2)
\end{align*}

Substituting this result into the formula, we get
\begin{align*}
  \frac{\partial F}{\partial x_1}(x_1, x_2)
  & = \int_{0}^{1}{\left[\frac{\dd }{\dd t}tf_1(tx_1, tx_2) + tx_2\left(\frac{\partial f_2}{\partial x_1}(tx_1, tx_2) - \frac{\partial f_1}{\partial x_2}(tx_1, tx_2) \right)\right] \;\mathrm{d}t}\\
  & = f_1(tx_1, tx_2)\big|_{t=0}^1\\
  & = f_1(x_1, x_2)
\end{align*}

Analogously, $\frac{\partial F }{\partial x_2} = f_2(x_1, x_2)$.
\end{proof}

Example \ref{example:1-2} and Theorem \ref{theorem:1-4} suggest that the answer to Question \ref{question:1-1} depends on the ``shape'' of
``topology'' of $U$. Instead of searching for a further examples or counterexamples of set $U$ and fucntion $f$, we
define an invariant of $U$, which tells us or not the question has an affirmative answer (for all $f$), assuming the 
necessary condition \eqref{eq:1-2}.

Give the open set $U\subset \RR^2$, let $C^\infty(U, \RR^k)$ denote the set of smooth functions $\phi:U\to\RR^k$. This 
is a vecter space. If $k=2$ one may consider $\phi:U\to\RR^k$ as a vecter filed on $U$ by plotting $\phi(u)$ from the
point $u$. We define the \Index{gradient} and \Index{rotation}:
\begin{align*}
  \grad: C^\infty(U, \RR) \to C^\infty(U, \RR^2), &&
  \grad: C^\infty(U, \RR^2) \to C^\infty(U, \RR) 
\end{align*}

by 
\begin{align*}
  \grad(\phi) = \left(\frac{\partial \phi}{\partial x_1}, \frac{\partial \phi}{\partial x_2}\right), &&
  \rot(\phi) = \frac{\partial \phi_1}{\partial x_2} - \frac{\partial \phi_2}{\partial x_1}
\end{align*}

Note that $\rot\circ \grad = 0$. Hence the kernel of rot contains the image of grad,
\begin{align*}
  \mathrm{Ker}(\mathrm{rot})&=\mathrm{Kernel~of~rot}\\
  \mathrm{Im}(\mathrm{grad})&=\mathrm{Image~of~grad}
\end{align*}

Since both rot and grad are linear operators, Im(grad) is a subspace of Ker(rot).
Therefore we can consider the quotient vector space, i.e. the vector space of
cosets 0: $\alpha$ + Im(grad) where 0: $\alpha\in$ Ker(rot):
\begin{align}
  H^1(U) = \ker(\rot)/\im(\grad).
\end{align}

Both Ker(rot) and Im(grad) are infinite-dimensional vector spaces. It is remarkable 
that the quotient space $H^1(U)$ is usually finite-dimensional. We can now reformulate 
Theorem \ref{theorem:1-4} as
\begin{align}
  H^1(U) = 0 \text{ where } U\subset \RR^2 \text{ is star-shaped}.
\end{align}

On the other hand, Example \ref{example:1-2} tells us that $H^1(\RR^2 - \{0\})\neq 0$. Later on we shall
see that $H^1(\RR^2 - \{0\})$ is 1-dimensional, and that $H^1(\RR^2 - \cup_{i=1}^k\{x_i\}) \cong \RR^k$. 
The dimension of $H^1(U)$ is the number of "holes" in $U$.

In analogy with (3) we introduce
\begin{align}
  H^0(U) = \ker(\grad)
\end{align}

This definition works for open sets $U$ of $\RR^k$ with $k\ge1$, when we define
\begin{align*}
  \grad(f) = \left(\frac{\partial f}{\partial x_1}, \ldots, \frac{\partial f}{\partial x_n}\right)
\end{align*}

\begin{theorem}\label{theorem:1-5}
  An open set $U\subset\RR^k$ is connected if and only if $H^0(U) = \RR$.
\end{theorem}

\begin{proof}
  Assume that $\grad(f) = 0$. Then f is locally constant: each $x_0\in U$ has
a neighborhood $V(x_0)$ with $f(x) = f(x_0)$ when $x\in V(x_0)$. If $U$ is connected,
then every locally constant function is constant. Indeed, for $x_0\in U$ the set
\begin{align*}
  \{x\in U|f(x) = f(x_0) = f^{-1}(f(x_0))\}
\end{align*}

is closed because $f$ is continuous, and open since $f$ is locally constant. Hence
it is equal to $U$, and $H^0(U) = \RR$. Conversely, if $U$ is not connected, then there
exists a smooth, surjective function $f:U\to \{0, 1\}$. Such a function is locally
constant, so $\grad(f) = 0$. It follows that $\dim H^0(U) > 1$.
\end{proof}

The reader may easily extend the proof of Theorem \ref{theorem:1-5} to show that $\dim H^0(U)$
is precisely the number of connected components of $U$.

\begin{example}
  Let $S = \{(x_1, x_2, x_3)\in \RR^3| x_1^2 + x_2^2 = 1, x_3 = 0\}$ be the unit circle
in the $(x_1, x_2)$-plane. Consider the function

\begin{align*}
  f(x_1,x_2,x_3)
  & = \left(\frac{-2x_1x_3}{x_3^2+\left(x_1^2+x_2^2-1\right)^2},\frac{-2x_2x_3}{x_3^2+\left(x_1^2+x_2^2-1\right)^2},\frac{x_1^2+x_2^2-1}{x_3^2+\left(x_1^2+x_2^2-1\right)^2}\right)
\end{align*}

on the open set $U = \RR^3 - S$.
\end{example}

One finds that $\rot(f) = 0$. Hence $f$ defines an element $[f] \in H^1(U)$. By
integration along a curve $\gamma$ in $U$, which is linked to $S$ (as two links in a chain),
we shall show that $[f] \neq 0$. The curve in question is
\begin{align*}
  \gamma(t) = \left(\sqrt{1+\cos t}, 0, \sin t\right), -\pi\le t\le \pi
\end{align*}

Assume $\grad(F) = f$ as a function on $U$. We can determine the integral of 
$\frac{\dd }{\dd t}F(\gamma(t))$ in two ways. On the hand we have
\begin{align*}
  \int_{\pi-\epsilon}^{-\pi+\epsilon}{\frac{\dd }{\dd t}F(\gamma(t)) \;\mathrm{d}t}
  = F(\gamma(-\pi+\epsilon)) - F(\gamma(\pi-\epsilon))
  \to 0\qquad \text{ for } \epsilon\to 0
\end{align*}

and on the other hand the chain rule gives
\begin{align*}
  \frac{d}{dt}F(\gamma(t))
  & = f_{1}(\gamma(t)) \cdot\gamma_{1}^{\prime}(t)+f_{2}(\gamma(t))\cdot\gamma_{2}^{\prime}(t)+f_{3}(\gamma(t))\cdot \gamma_{3}^{\prime}(t) \\
  & = \sin^{2}t+0+\cos^{2}\mathrm{t}=1.
\end{align*}

Therefore the integral also converges to $2\pi$, which is a contradiction.

\begin{example}
  Let $U$ be an open set in $\RR^k$ and $X: U\to \RR^k$ a smooth function (a smooth vector field). 
Recall that the \Index{energy} $A_\gamma(X)$, of $X$ along a smooth curve $\gamma: [a, b]\to U$ is 
defined by the integral
\begin{align*}
  A_\gamma(X) = \int_{a}^{b}{\langle X\circ \gamma(t), \gamma'(t)\rangle\;\mathrm{d}t}
\end{align*}

where $\langle\cdot\rangle$ denotes the standard product. If $X = \grad(\Phi)$ and $\Phi_\gamma(a) = \Phi_\gamma(b)$,
then the energy is zero, since 
\begin{align*}
  \langle X\circ \gamma(t), \gamma'(t)\rangle = \frac{\dd }{\dd t}\Phi(\gamma(t))
\end{align*}

by the rule; compare Example \ref{example:1-2}.
\end{example}